\documentclass[10pt,a4paper]{report}
\usepackage[latin1]{inputenc}
\usepackage[english]{babel}
\usepackage{amsmath}
\usepackage{amsfonts}
\usepackage{amssymb}
\usepackage{graphicx}
\usepackage{pgfgantt}
\usepackage{rotating}
\usepackage{tabularx}
\usepackage[a4paper]{geometry}
\usepackage[nottoc]{tocbibind}
\usepackage{url} % hyperref works too
\urlstyle{sf}  % (sf also works, for something more subtle than tt)

\setlength{\overfullrule}{0pt}
\newcommand{\HRule}{\rule{\linewidth}{0.5mm}}

\title{Natural Language Processing with Machine Learning\\Preliminary Report}
\author{Luis Nieto Pi\~na}
\date{January 18, 2013}
\makeindex
\begin{document}
\begin{titlepage}

\begin{center}

	% Upper part of the page
	
	\includegraphics[width=4cm,height=1.5cm]{uab}
    \hfill%
    \includegraphics[height=1.5cm]{ee}
    \vspace*{3cm}
	    
	
	\textsc{\LARGE ENGINYERIA INFORM\`ATICA}\\[1.5cm]
	
	\textsc{\Large MEM\`ORIA PR\`EVIA DEL PROJECTE}\\[0.5cm]
	
	\HRule \\[0.4cm]
	
	\textsc{\huge \bfseries 5322 \\ Natural Language Processing with Machine Learning}\\[0.5cm]
	
	\HRule \\
	
	\vfill
	
	% Bottom of the page
	\begin{tabularx}{\textwidth}{ |X|X| }
		\hline
		  Signatura de l'estudiant\newline\newline\newline\newline\newline\newline\newline Nom: Luis Nieto Pi\~na\newline\newline Data: 18 de gener de 2013 & Signatura del director/a o directors/es\newline\newline\newline\newline Nom/s: Mar\c{c}al Rusi\~nol Sanabra\newline\newline Dpt: Ci\`encies de de la Computaci\'o \newline\newline Data: 18 de gener de 2013 \\
		\hline
	\end{tabularx}
	

\end{center}

\end{titlepage}
\tableofcontents

\chapter{Project's goals}
	
	Sentiment Analysis is the part of natural language processing that deals with identifying the sentiment of the author of a text; i.e., recognizing whether or not the text is opinionated and if so, whether it has a strong bias towards or against some predefined topic.\\
	
	As an example of opinionated texts, in many Internet-shops, e.g. Amazon, it is possible for the customer to leave feedback about an item. These reviews may sometimes be accompanied (?) with an additional rating system such as stars (see Fig. 1), which may be considered an additional sentiment data source. This feedback reviews come in many different forms: long, short, detailed, a list of keywords or even a story about an experience the customer had. , other texts appearing in the media (be it social media or classical media like newspaper articles) are often complex and not easily reducible to a simple star-rating.\\
	
	% PROBLEMS WITH DIFFERENT TEXT FORMATS AND THEIR CHARACTERISTICS (fix above)	
	
	To analyse the sentiment of the author, different approaches are possible. A straight-forward way would be having  listings of positive words and negative words and simply count their occurrence. More sophisticated approaches learn autonomously, given a set of texts and their manually assigned sentiment label, which words or phrases are important.\\

	In this project we will try to achieve an alternative approach to the task of sentiment analysis. Traditionally, most sentiment analysis approaches rely on a heavy preprocessing of the texts and selecting features in some more or less automated way. Our aim is to see whether a decrease of human intervention and intuition is possible with aid of some well-known machine learning tools, namely neural networks (NN), and compare its performance against that of consolidated techniques used for sentiment analysis, such as latent semantic analysis (LSA).\\
	
	In detail, the goals which we intend to achieve are:
	
	\begin{enumerate}
		\item Get an adequate set of texts (\textit{corpus}). This dataset should be comprised of opinionated texts within at least one field of opinion; i.e., texts containing opinions on a similar topic.
		
		\item Carry out a preprocessing task of the dataset will have to be done in order to adapt it to the project's specifications.
		
		\item Implement an LSA application of sentiment analysis and test its performance against current standards. This performance will represent the project's baseline.
		
		\item Implement a NN application for sentiment analysis and test its performance against that of the LSA implementation. This is intended to be the main task of the project and its reach will be adapted to the development of its results. It would be desirable to test more than one implementation of neural networks.
		
	\end{enumerate}
	
\chapter{Estate of the art}	
\chapter{Feasibility}

	\section{Dataset}

	This project's implementation is relies largely on the database on top of which the applications will be tested: its size will play key role for the applications' training and thus their reliability and its contents need to be fit for this task; i.e., a set of opinionated texts on a clearly defined topic (consumer reviews for positive/negative sentiment, political articles for left-wing/right-wing sentiment, etc.).\\
	
	Some examples of such corpora are MPQA Opinion Corpus \cite{mpqa}, which comprises 692 annotated documents (annotations include attitude: positive sentiment, positive arguing, positive intention, etc.) from newspapers, transcriptions of conversations, reports, etc., and Darmstadt Service Review Corpus \cite{dsrc} which, as announced on its webpage, should be composed of sentence-annotated consumer reviews from two different websites.
	
	\section{Users}
	
		Different system's user profiles are considered:\\
	
		\begin{tabularx}{\textwidth}{|X|X|}
			\hline
			\textbf{User} & \textbf{Capabilities} \\
			\hline
			Expert user & He/She is able to configure the system; e.g., set the applications to accept a new format of datasets, change the desired output of the system. \\
			\hline
			Standard user & This user is capable of feed new input (on a specified format) to the system and read/interpret the resulting output. \\
			\hline
		\end{tabularx}
		
	\section{Development team}
	
		The required team for the development of this project is:\\
		
		\begin{tabularx}{\textwidth}{|X|X|}
			\hline
			\textbf{Role} & \textbf{Task} \\
			\hline
			Project Administrator & Defines, manages and plans the project. \\
			\hline
			Project Director & Supervises the Project Administrator's task developement. \\
			\hline
			Main Developer & Designs and implements the applications; his/her tasks are those of analist and programmer. \\
			\hline
			Testing Technician & Develops test sets to check the applications correct behavior and the achievement of all requirements. \\
			\hline
		\end{tabularx}

	\section{Functional requirements}
		\begin{itemize}
			\item Normalization (preprocessing) of a set of text files.
			\item Input/output should be both configurable.
		\end{itemize}
	\section{Non-functional requirements}
		\begin{itemize}
			\item Time performance. (!)
			\item Open source. (!)
			\item Programming language restrictions. (!)
		\end{itemize}
	\section{System restrictions}
		\begin{itemize}
			\item The applications will be implemented on a UNIX system.
			\item HW requirements (!)
			\item The project's documentation (and thus the project itself) must be handed in before june 19th, 2013.
		\end{itemize}
\chapter{Planning}

	The duration of this project's development is 9 months with an approximated dedication from 2 to 4 hours per day, depending on the task. The project is divided in six groups of tasks:
	
	\begin{enumerate}
		\item \textbf{Preparation} (8 weeks). Familiarize with some existing tools and previous work in order to determine the best approach to the development task; research corpora for use as datasets.
		\begin{enumerate}
			\item[1.1] WordNet/NLTK tools for preprocessing of texts.
			\item[1.2] LSA techniques and existing implementations.
			\item[1.3] Research on different implementations of neural networks.
			\item[1.4] Search for datasets.
		\end{enumerate}
		\item \textbf{Preliminary report} (4 weeks). Synthesize information and write this report.
		\item \textbf{Text preprocessing} (4 weeks)
		\begin{enumerate}
			\item[3.1] Implementation of text preprocessing tools.
			\item[3.2] Testing.
		\end{enumerate}
		\item \textbf{Latent Semantic Analysis} (4 weeks)
		\begin{enumerate}
			\item[4.1] Implementation of LSA application.
			\item[4.2] Testing.
		\end{enumerate}
		\item \textbf{Neural Networks} (8 weeks)
		\begin{enumerate}
			\item[5.1] Implementation of neural networks.
			\item[5.2] Testing.
		\end{enumerate}
		\item \textbf{Final report} (15 weeks). Write the final report. This task overlaps the development tasks as it is intended to be written gradually: a draft of every section should be kept while developing and be completed soon after finishing each corresponding task. Some time will be needed after finishing the development for summarizing and concluding the report.
		\begin{enumerate}
			\item[6.1] Text preprocessing.
			\item[6.2] LSA.
			\item[6.3] NN.
			\item[6.4] Summarize.
		\end{enumerate}
	\end{enumerate}
	\newgeometry{top=2.5cm}
	\begin{figure}[ftbp]
	\begin{center}
	\begin{sideways}
		\begin{ganttchart}[y unit title=0.4cm,
		y unit chart=0.5cm,
		vgrid={draw=none, dotted},
		hgrid, 
		title label anchor/.style={below=-1.6ex},
		title left shift=.05,
		title right shift=-.05,
		title height=1,
		bar/.style={fill=gray!50},
		incomplete/.style={fill=white},
		progress label text={},
		bar height=0.7,
		group right shift=0,
		group top shift=.6,
		group height=.3,
		group peaks={}{}{.2}]{36}
		
		%labels
		\gantttitle{2012}{12}
		\gantttitle{2013}{24} \\
		\gantttitle{October}{4}
		\gantttitle{November}{4} 
		\gantttitle{December}{4} 
		\gantttitle{January}{4} 
		\gantttitle{February}{4} 
		\gantttitle{March}{4} 
		\gantttitle{April}{4}
		\gantttitle{May}{4} 
		\gantttitle{June}{4} \\
		
		%tasks	
		\ganttgroup{Preparation}{1}{8}\\	% 0
		\ganttbar{WordNet/NLTK}{1}{2} \\ 
		\ganttbar{Existing LSA tools}{3}{4} \\ 
		\ganttbar{Related NN works}{5}{6}\\ 
		\ganttbar{Search for data sources}{1}{8} \\ 
		
		\ganttgroup{Preliminary report}{9}{12} \\ 	% 5
		\ganttbar{Write previous report}{9}{12} \\ 
		
		\ganttmilestone{Submit preliminary report}{15} \\
		
		\ganttgroup{Text preprocessing}{13}{16} \\	%7
		\ganttbar{Write routines for filtering}{13}{14} \\ 
		\ganttbar{Test routines}{15}{16} \\
		
		\ganttgroup{Latent Semantic Analysis}{17}{20} \\ % 10
		\ganttbar{Implement LSA}{17}{18} \\
		\ganttbar{Test LSA}{19}{20} \\
		
		\ganttgroup{Neural Networks}{21}{28} \\ %13
		\ganttbar{Implement NN}{21}{24} \\
		\ganttbar{Test NN}{25}{28} \\
		
		\ganttmilestone[milestone yshift=.6]{Request for presentation}{32} \\
		
		\ganttgroup{Final report}{17}{35} \\ %16
		\ganttbar{Text preprocessing}{17}{20} \\
		\ganttbar{Latent Semantic Analysis}{21}{24} \\
		\ganttbar{Neural Networks}{29}{32}\\
		\ganttbar{Summarize}{33}{35} \\
		\ganttmilestone{Submit final report}{35} \\
		
		\ganttmilestone[milestone yshift=.6]{Presentation}{36}
			
		%relations 
		\ganttlink{elem5}{elem7}
		
		\ganttlink{elem9}{elem10}
		\ganttlink{elem12}{elem13}
		\ganttlink{elem15}{elem16}
		
		\ganttlink{elem8}{elem19}
		\ganttlink{elem11}{elem20}
		\ganttlink{elem14}{elem21}
		\ganttlink{elem18}{elem23}
		
		\ganttlink{elem17}{elem24}
		
		\end{ganttchart}
	\end{sideways}
	\end{center}
	\end{figure}
	\restoregeometry
	
	%BIBLIOGRAPHY
	\begin{thebibliography}{9}

	\bibitem{mpqa}
		\emph{MPQA Opinion Corpus} \url{http://www.cs.pitt.edu/mpqa/mpqa_corpus.html}.
		
	\bibitem{dsrc}
		\emph{Darmstadt Service Review Corpus} \url{http://www.ukp.tu-darmstadt.de/data/sentiment-analysis/darmstadt-service-review-corpus/}.
	
	\end{thebibliography}

\end{document}