\documentclass[10pt,a4paper,draft]{report}
\usepackage[latin1]{inputenc}
\usepackage[english]{babel}
\usepackage{amsmath}
\usepackage{amsfonts}
\usepackage{amssymb}
\usepackage{graphicx}

\title{Natural Language Processing with Machine Learning\\Preliminary Report}
\author{Luis Nieto Pi\~na}
\date{January 18, 2013}

\begin{document}
\maketitle
\makeindex

\begin{abstract}
		The task of processing text input given in natural language is quite challenging for computers. As opposed to structured data or even source code, the way humans speak is highly complex. Different phrases may have the same meaning, while at the same time the same phrase may have different meanings, depending upon the context or even the way some words are pronounced. Yet, for humans, interacting with a computer is most convenient if they can use natural language. In this PFC, methods from machine learning should be used for sentiment analysis, i.e. identifying the opinion of the writer towards the subject of the text.
\end{abstract}

\chapter{Introduction}
	Sentiment Analysis is the part of natural language processing that deals with identifying the sentiment of the author of a text, i.e., recognizing whether or not the text is opinionated and if so, whether the bias is (strongly) towards or against the main talking point. Many Internet-Shops, e.g. Amazon, it is possible for the customer to write a feedback about the article. Obviously this feedback can be of any form: long, short, detailed, a list of keywords or even a story about an experience the customer had with the article.\\

	Sometimes, an additional rating systems, e.g., with stars (see Fig. 1), sometimes not. In addition, texts appearing in the media (may it be social media like an entry in Facebook or classical media like a newspaper article) are often complex and not easily reducible to a simple star-rating.\\

	To analyze the sentiment of the author, different approaches are possible. A straight-forward way would be to have a list of positive words and negative words and simply count their occurrence. More sophisticated approaches learn autonomously, given a set of texts and their manually assigned sentiment label, which words or phrases are important. In this PFC, machine learning techniques (mainly neural networks and latent semantic analysis (LSA))
should be implemented and tested for their applicability for the given task.
\chapter{State of the art}
\chapter{Project's goals}
\chapter{Requirements}
\chapter{Feasibility}
\chapter{Planning}

\end{document}